\section{Decentralized Training Strategies}

% All this section will be 1/2 pages.
During centralized \gls*{gan} training, the generator takes noise as input and produces fake data. The discriminator tries to classify it as fake or real. If the discriminator is successful, it provides feedback to the generator to help it improve its data generation. This process continues until the generator is able to produce data that is realistic enough to fool the discriminator.

In a decentralized setting, there is one generator and multiple discriminators. 
At the server side, the generator takes a random Gaussian noise as input and outputs fake samples. The fake samples are sent to each worker. The local discriminator at each worker is trained on both the local real samples and the fake samples received from the server. Two losses are calculated, the loss on real samples (Real Loss) and the loss on fake samples (Fake Loss). 
%
The fake loss is expected to be high when the discriminator is deceived by the fake samples. Conversely, if the fake loss is low, it means the discriminator was able to classify the fake sample correctly. The fake losses from all workers are compared to each other. We examined three possible strategies to compare fake losses. 

If we choose the worker with the highest fake loss, this would be the discriminator that was most deceived by the fake samples, which could be called \emph{most forgiving}. If we select the worker with the lowest fake loss, this corresponds to the discriminator that was most successful at distinguishing fake from real samples. This discriminator could be considered the \emph{least forgiving} or the most formidable adversary. The generator at the server side is trained against the chosen discriminator using back-propagation, whether it is the \emph{most} or \emph{least forgiving}. The break down of the architecture and logic for both strategies are shown in Fig.~\ref{fig:f2u_l}.
%
\def\ftouscale{0.6}
\begin{figure}[t]
\centering
\begin{tikzpicture}[mylabel/.style={anchor=north west,font=\bfseries},
mylabel1/.style={anchor=north west,xshift=0.4 cm,font=\bfseries},font=\normalsize,scale=\ftouscale,every node/.style={scale=\ftouscale,rounded corners},line width = 0.6pt]

\newcommand\workwidth{110}
\newcommand\workheight{50}
\newcommand\workdist{2.4}

\newcommand\servheight{3.0cm}
\newcommand\genx{-0.3}

    \node[draw,rectangle,rounded corners,minimum width=4cm,minimum height=\servheight,fill=servcol, label={[mylabel]north west:Server}] (serv) at (-1.1,0) {};
    
    \node[draw,rectangle,rounded corners,fill=white] (merg) at (\genx,-\servheight/2) {Feed Back};
    
    \node[draw,rectangle,rounded corners,minimum width=\workwidth,minimum height=\workheight,fill=workcol, label={[mylabel1]Worker}] (work1) at (3.2,\workdist) {}; 
    \node[draw,rectangle,rounded corners,minimum width=\workwidth,minimum height=\workheight,fill=workcol,label={[mylabel1]Worker}] (work2) at (3.2,0) {}; 
    \node[draw,rectangle,rounded corners,minimum width=\workwidth,minimum height=\workheight,fill=workcol,label={[mylabel1]Worker}] (work3) at (3.2,-\workdist) {}; 
    
    \node[draw,rectangle,fill=white] (gen) at ( \genx,-0.4) {Generator};
    \node[draw,rectangle,fill=white] (noise) at ( \genx-2.0,-0.4) {Noise};
    
\newcommand\discx{3.4cm}
\newcommand\discy{1.9cm}
\newcommand\discydis{2.4cm}

    \node[draw,rectangle,fill=white] (disc1) at (\discx,\discy) {Discriminator};
    \node[draw,rectangle,fill=white] (disc2) at (\discx,\discy-\discydis) {Discriminator};
    \node[draw,rectangle,fill=white] (disc3) at (\discx,\discy-\discydis*2) {Discriminator};
    
\newcommand\fakesize{0.6cm}
\newcommand\fakey{0.9}
\newcommand\fakedist{0.1}

    \node[draw,align=center,minimum size=\fakesize,fill=white,sharp corners] (fake1) at (\genx,\fakey) {};
    \node[draw,align=center,minimum size=\fakesize,fill=white,sharp corners] (fake2) at (\genx+\fakedist,\fakey-\fakedist) {};
    \node[draw,align=center,minimum size=\fakesize,fill=white,sharp corners] (fake3) at (\genx+\fakedist*2,\fakey-\fakedist*2) {};

\newcommand\dbwidth{5mm}
\newcommand\dbheight{5mm}
\newcommand\dbx{1.7}
\newcommand\dby{2.8}
\newcommand\dbydist{2.4}

    \node[cylinder,draw=black,aspect = 0.5,shape border rotate = 90,minimum width = \dbwidth, minimum height = \dbheight,fill=white,sharp corners] (db1) at (\dbx,\dby) {};    
    \node[cylinder,draw=black,aspect = 0.5,shape border rotate = 90,minimum width = \dbwidth, minimum height = \dbheight,fill=white,sharp corners] (db2) at (\dbx,\dby-\dbydist) {}; 
    \node[cylinder,draw=black,aspect = 0.5,shape border rotate = 90,minimum width = \dbwidth, minimum height = \dbheight,fill=white,sharp corners] (db3) at (\dbx,\dby-\dbydist*2) {};
    
    \draw[] (noise) -- (gen);
    \draw[] (merg) -- (gen);
    \draw[] (gen.north) -- (fake3.237);    


    \draw[rounded corners] (db1.south) |-  (disc1.175); 
    \draw[rounded corners] (db2.south) |-  (disc2.175); 
    \draw[rounded corners] (db3.south) |-  (disc3.175); 
    
    \newcommand\arrowadjust{5.6mm}
    
    % \draw[rounded corners] (fake1.93) -- + (0,8mm) -| + (14mm,\arrowadjust) --  (disc1.185); 
    \draw[rounded corners] (fake1.93) |- (disc1.185);
    
    \draw[] (1.084,\fakey + \fakesize *2 + \arrowadjust) |-  (disc2.185);
    \draw[rounded corners] (1.084,\fakey + \fakesize *2 + \arrowadjust-\discydis) |-  (disc3.185); 

\newcommand\lossxdist{3.5cm}
\newcommand\lossydist{0.7cm}
\newcommand\disclossx{1.8}
\newcommand\fakelossdist{2.9}

\node[draw,rectangle,fill=white] (reall1) at (\discx+\lossxdist,\discy+\lossydist) {Real Loss};
\node[draw,rectangle,fill=fakecol] (fakel1) at (\discx+\lossxdist,\discy-\fakelossdist) {Fake Loss};

\node[draw,rectangle,fill=white] (reall2) at (\discx+\lossxdist,\discy-\discydis+\lossydist) {Real Loss};
\node[draw,rectangle,fill=fakecol] (fakel2) at (\discx+\lossxdist,\discy-\discydis-\fakelossdist) {Fake Loss};

\node[draw,rectangle,fill=white] (reall3) at (\discx+\lossxdist,\discy-\discydis*2+\lossydist) {Real Loss};
\node[draw,rectangle,fill=fakecol] (fakel3) at (\discx+\lossxdist,\discy-\discydis*2-\fakelossdist) {Fake Loss};

\node[draw,rectangle,fill=white,align=center] (avg1) at (\discx+\lossxdist*\disclossx,\discy+\lossydist/2) {Discriminator \\Loss (avg)};

\node[draw,rectangle,fill=white,align=center] (avg2) at (\discx+\lossxdist*\disclossx,\discy+\lossydist/2-\discydis) {Discriminator\\ Loss (avg)};

\node[draw,rectangle,fill=white,align=center] (avg3) at (\discx+\lossxdist*\disclossx,\discy+\lossydist/2-\discydis*2) {Discriminator\\ Loss (avg)};


\draw[rounded corners=2 pt] (disc1.5) -- + (10mm,0) |- (reall1.west);
\draw[] (disc1.-5) -- (fakel1.west);
\draw[rounded corners=2 pt] (reall1.5) -- + (3mm,0) |- (avg1.175);
\draw[rounded corners=2 pt] (fakel1.east) -- + (3mm,0) |- (avg1.185);
\draw[rounded corners] (avg1.north) -- + (0,7mm) -|  (disc1.north);

\draw[rounded corners=2 pt] (disc2.5) -- + (10mm,0) |- (reall2.west);
\draw[] (disc2.-5) -- (fakel2.west);
\draw[rounded corners=2 pt] (reall2.5) -- + (3mm,0) |- (avg2.175);
\draw[rounded corners=2 pt] (fakel2.east) -- + (3mm,0) |- (avg2.185);
\draw[rounded corners=2 pt] (avg2.north) -- + (0,7mm) -|  (disc2.north);

\draw[rounded corners=2 pt] (disc3.5) -- + (10mm,0) |- (reall3.west);
\draw[] (disc3.-5) -- (fakel3.west);
\draw[rounded corners=2 pt] (reall3.5) -- + (3mm,0) |- (avg3.175);
\draw[rounded corners=2 pt] (fakel3.east) -- + (3mm,0) |- (avg3.185);
\draw[rounded corners=2 pt] (avg3.north) -- + (0,7mm) -|  (disc3.north);


\newcommand\choosedist{0.5}
\newcommand\choosedistone{1.2}
\node[draw,rectangle,fill=tablecolfour,align=center,minimum height = 1cm] (choose) at (3.5,\choosedist-\workdist*2) {choose worker with \\ highest fake loss};
\node[draw,rectangle,fill=tablecolone,align=center,minimum height = 1cm] (choose1) at (3.5,\choosedist-\workdist*2-\choosedistone) {choose worker with \\lowest fake loss};

\node[draw,rectangle,fill=tablecolfour,align=center, minimum height = 1cm, minimum width = 2.8cm] (mostforg) at (\genx,\choosedist-\workdist*2) {Most Forgiving \\ Discriminator};

\node[draw,rectangle,fill=tablecolone,align=center, minimum height = 1cm, minimum width = 2.8cm] (leastforg) at (\genx,\choosedist-\workdist*2-\choosedistone) {Least Forgiving \\ Discriminator};

\draw[color=myred,rounded corners] (fakel3.south) |- (choose.7);
\draw[color=myred,rounded corners] (fakel2.south) -- + (0,-2mm) -| + (41.5mm,-30.5mm) |- (choose.east); 
\draw[color=myred,rounded corners] (fakel1.south) -- + (0,-2mm) -| + (43.5mm,-55mm) |- (choose.-7); 


% empty patches
\node[fill=white,rectangle,scale=0.5,above right = -0.37cm and 0.5cm of choose,anchor=west,minimum height=0.6cm,sharp corners] (emp1) {};

\node[fill=white,rectangle,scale=0.5,above right = -0.43cm and 0.63cm of choose,anchor=west,sharp corners] (emp2) {};


\draw[color=myred,rounded corners] (5.95,\choosedist-\workdist*2+0.19) |- (choose1.7);
\draw[color=myred,rounded corners] (6.15,\choosedist-\workdist*2) |- (choose1.east);
\draw[color=myred,rounded corners] (6.35,\choosedist-\workdist*2-0.19) |- (choose1.-7);

\draw[] (choose.west) -- (mostforg.east);
\draw[] (choose1.west) -- (leastforg.east);

% \draw[] (mostforg.north) -- (merg.south);    
\draw[rounded corners] (mostforg.west) -- + (-4.5mm,0mm) |- + (6mm,13mm) -| (merg.295);
\draw[rounded corners] (leastforg.west) -- + (-6.5mm,0mm) |- + (7mm,27mm) -| (merg.255);

\end{tikzpicture}
\caption{Comparison of the most and least forgiving networks training process.}
\label{fig:f2u_l}
\end{figure}

The third strategy is a \emph{weighted average}. In this setting, the discriminator parameters from each worker are all aggregated to form a global discriminator model. In order to emphasize the parameters of the discriminator with the highest loss (\emph{most forgiving}), we use a weight that reflects the calculated loss. This can be achieved by employing a Softmax function as shown in Fig.~\ref{fig:f2a_arc}. The Softmax function ensures that the output is a probability distribution, where each element is non-negative and the sum of all the elements is 1. The averaging process is shown in Alg.~\ref{alg:weighted_fed_avg}. The algorithm described is a modified Federated Average algorithm, which emphasizes the \emph{most forgiving} discriminator. However, we could easily switch the behaviour to favoring the \emph{least forgiving} (lowest loss) by multiplying fake losses by $-1$ before applying the Softmax.

% \atB{Is there any novelty wrt the original approach? If not, I would remove the algorithm and explain it in the text.} 

\begin{figure}[t]
\centering
\begin{tikzpicture}[mylabel/.style={anchor=north west,font=\bfseries},
mylabel1/.style={anchor=north west,xshift=0.4 cm,font=\bfseries},font=\normalsize,scale=\ftouscale,every node/.style={scale=\ftouscale,rounded corners},line width = 0.6pt]

\newcommand\workwidth{110}
\newcommand\workheight{50}
\newcommand\workdist{2.4}

\newcommand\servheight{3.0cm}
\newcommand\genx{-0.3}

    \node[draw,rectangle,rounded corners,minimum width=4cm,minimum height=\servheight,fill=servcol, label={[mylabel]north west:Server}] (serv) at (-1.1,0) {};
    
    \node[draw,rectangle,rounded corners,fill=white] (merg) at (\genx,-\servheight/2) {Feed Back};
    
    \node[draw,rectangle,rounded corners,minimum width=\workwidth,minimum height=\workheight,fill=workcol, label={[mylabel1]Worker}] (work1) at (3.2,\workdist) {}; 
    \node[draw,rectangle,rounded corners,minimum width=\workwidth,minimum height=\workheight,fill=workcol,label={[mylabel1]Worker}] (work2) at (3.2,0) {}; 
    \node[draw,rectangle,rounded corners,minimum width=\workwidth,minimum height=\workheight,fill=workcol,label={[mylabel1]Worker}] (work3) at (3.2,-\workdist) {}; 
    
    \node[draw,rectangle,fill=white] (gen) at ( \genx,-0.4) {Generator};
    \node[draw,rectangle,fill=white] (noise) at ( \genx-2.0,-0.4) {Noise};
    
\newcommand\discx{3.4cm}
\newcommand\discy{1.9cm}
\newcommand\discydis{2.4cm}

    \node[draw,rectangle,fill=white] (disc1) at (\discx,\discy) {Discriminator};
    \node[draw,rectangle,fill=white] (disc2) at (\discx,\discy-\discydis) {Discriminator};
    \node[draw,rectangle,fill=white] (disc3) at (\discx,\discy-\discydis*2) {Discriminator};
    
\newcommand\fakesize{0.6cm}
\newcommand\fakey{0.9}
\newcommand\fakedist{0.1}

    \node[draw,sharp corners,align=center,minimum size=\fakesize,fill=white] (fake1) at (\genx,\fakey) {};
    \node[draw,sharp corners,align=center,minimum size=\fakesize,fill=white] (fake2) at (\genx+\fakedist,\fakey-\fakedist) {};
    \node[draw,sharp corners,align=center,minimum size=\fakesize,fill=white] (fake3) at (\genx+\fakedist*2,\fakey-\fakedist*2) {};

\newcommand\dbwidth{5mm}
\newcommand\dbheight{5mm}
\newcommand\dbx{1.7}
\newcommand\dby{2.8}
\newcommand\dbydist{2.4}

    \node[cylinder,sharp corners,draw=black,aspect = 0.5,shape border rotate = 90,minimum width = \dbwidth, minimum height = \dbheight,fill=white] (db1) at (\dbx,\dby) {};    
    \node[cylinder,sharp corners,draw=black,aspect = 0.5,shape border rotate = 90,minimum width = \dbwidth, minimum height = \dbheight,fill=white] (db2) at (\dbx,\dby-\dbydist) {}; 
    \node[cylinder,sharp corners,draw=black,aspect = 0.5,shape border rotate = 90,minimum width = \dbwidth, minimum height = \dbheight,fill=white] (db3) at (\dbx,\dby-\dbydist*2) {};
    
    \draw[] (noise) -- (gen);
    \draw[] (merg) -- (gen);
    \draw[] (gen.north) -- (fake3.237);    
    % \draw[] (disc1.-5) -- + (8mm,0) |- + (0,-3.5cm) -| (merg.230);
    % \draw[] (disc2.-5) -- + (7mm,0) |- + (0,-2.0cm) -| (merg.south);
    % \draw[] (disc3.-5) -- + (6mm,0) |- + (0,-0.5cm) -| (merg.310);

    \draw[rounded corners] (db1.south) |-  (disc1.175); 
    \draw[rounded corners] (db2.south) |-  (disc2.175); 
    \draw[rounded corners] (db3.south) |-  (disc3.175); 
    
    \newcommand\arrowadjust{5.7mm}
    
    \draw[rounded corners] (fake1.93) |-  (disc1.185); 
    \draw[] (1.084,\fakey + \fakesize *2 + \arrowadjust) |-  (disc2.185);
    \draw[rounded corners] (1.084,\fakey + \fakesize *2 + \arrowadjust-\discydis) |-  (disc3.185); 

\newcommand\lossxdist{3.5cm}
\newcommand\lossydist{0.7cm}
\newcommand\disclossx{1.8}
\newcommand\fakelossdist{2.9}

\node[draw,rectangle,fill=white] (reall1) at (\discx+\lossxdist,\discy+\lossydist) {Real Loss};
\node[draw,rectangle,fill=fakecol] (fakel1) at (\discx+\lossxdist,\discy-\fakelossdist) {Fake Loss};

\node[draw,rectangle,fill=white] (reall2) at (\discx+\lossxdist,\discy-\discydis+\lossydist) {Real Loss};
\node[draw,rectangle,fill=fakecol] (fakel2) at (\discx+\lossxdist,\discy-\discydis-\fakelossdist) {Fake Loss};

\node[draw,rectangle,fill=white] (reall3) at (\discx+\lossxdist,\discy-\discydis*2+\lossydist) {Real Loss};
\node[draw,rectangle,fill=fakecol] (fakel3) at (\discx+\lossxdist,\discy-\discydis*2-\fakelossdist) {Fake Loss};

\node[draw,rectangle,fill=white,align=center] (avg1) at (\discx+\lossxdist*\disclossx,\discy+\lossydist/2) {Discriminator \\Loss (avg)};

\node[draw,rectangle,fill=white,align=center] (avg2) at (\discx+\lossxdist*\disclossx,\discy+\lossydist/2-\discydis) {Discriminator \\Loss (avg)};

\node[draw,rectangle,fill=white,align=center] (avg3) at (\discx+\lossxdist*\disclossx,\discy+\lossydist/2-\discydis*2) {Discriminator \\Loss (avg)};


\draw[rounded corners =2 pt] (disc1.5) -- + (10mm,0) |- (reall1.west);
\draw[] (disc1.-5) -- (fakel1.west);
\draw[rounded corners=2 pt] (reall1.5) -- + (3mm,0) |- (avg1.175);
\draw[rounded corners=2 pt] (fakel1.east) -- + (3mm,0) |- (avg1.185);
\draw[rounded corners=2 pt] (avg1.north) -- + (0,7mm) -|  (disc1.north);

\draw[rounded corners=2 pt] (disc2.5) -- + (10mm,0) |- (reall2.west);
\draw[] (disc2.-5) -- (fakel2.west);
\draw[rounded corners=2 pt] (reall2.5) -- + (3mm,0) |- (avg2.175);
\draw[rounded corners=2 pt] (fakel2.east) -- + (3mm,0) |- (avg2.185);
\draw[rounded corners=2 pt] (avg2.north) -- + (0,7mm) -|  (disc2.north);

\draw[rounded corners=2 pt] (disc3.5) -- + (10mm,0) |- (reall3.west);
\draw[] (disc3.-5) -- (fakel3.west);
\draw[rounded corners=2 pt] (reall3.5) -- + (3mm,0) |- (avg3.175);
\draw[rounded corners=2 pt] (fakel3.east) -- + (3mm,0) |- (avg3.185);
\draw[rounded corners=2 pt] (avg3.north) -- + (0,7mm) -|  (disc3.north);


\newcommand\choosedist{0.1}
\newcommand\choosedistone{1}
\node[draw,rectangle,fill=bluelight,minimum height=1cm] (softmax) at (5.7,\choosedist-\workdist*2) {SoftMax};

\node[draw,rectangle,fill=bluelight,align=center] (wavg) at (2.3,\choosedist-\workdist*2) {Weighted \\ Average};

\node[draw,rectangle,fill=bluelight,align=center] (gdisc) at (\genx,\choosedist-\workdist*2) {Global \\ Discriminator};

\draw[color=myred,rounded corners] (fakel3.280) |- (softmax.15);
\draw[color=myred,rounded corners] (fakel2.south) -- + (0,-2mm) -| + (41.5mm,-34.5mm) |- (softmax.east); 
\draw[color=myred,rounded corners] (fakel1.south) -- + (0,-2mm) -| + (43.5mm,-60mm) |- (softmax.-15); 


\draw[] (softmax.west) -- node[above,font=\small,align=center,pos=0.45] {model\\ contribution} (wavg.east);
\draw[] (wavg.west) -- (gdisc.east);
\draw[] (gdisc.north) -- (merg.south);

\draw[line width = 1pt, color=myred,densely dashed]  (disc1.200) to [bend right=20]  node[sloped, above] {}  (wavg.120) ;
\draw[line width = 1pt, color=myred,densely dashed]  (disc2.200) to [bend right=20]  node[sloped, above] {}  (wavg.90) ;
\draw[line width = 1pt, color=myred,densely dashed]  (disc3.200) to [bend right=20]  node[below,pos=0.5,anchor=west,font = \small] {Discriminator Parameters}  (wavg.60) ;

% \draw[] (mostforg.north) -- (merg.south);    
% \draw[] (wavg.west) -- + (-3mm,0mm) |- + (6mm,12mm) -| (merg.300);


\end{tikzpicture}
\caption{Schematic of the F2A network using a Softmax function.}
\label{fig:f2a_arc}
\end{figure}



\begin{algorithm}[t]
\caption{Weighted Federated Averaging [F2A]}
\label{alg:weighted_fed_avg}
\textbf{ Input:} model\_weights $w_m$, contribution $c$
\begin{algorithmic}[1]
%\Function{weighted\_fed\_average}{, }
    \State $w_{\mathrm{avg}}$ $\gets$ $w_m[0, :]$ \Comment{First row of $w_m$}
    \For{$j$ $\gets$ 0 to \texttt{len}($w_{\mathrm{avg}}$)}
        \State $t_w \gets$ \texttt{zeros\_like}($w_m[0, j]$)
        \For{$i \gets 0$ to \texttt{rows}($w_m$)} \Comment{Loop over rows of $w_m$}
            \State $t_w \gets t_w + c[i]  w_m[i, j]$
        \EndFor
        \State $w_{\mathrm{avg}}[j] \gets t_w$
    \EndFor
    \State \Return w\_avg
%\EndFunction
\end{algorithmic}
\end{algorithm}



\subsection{Data-set Pre-processing}
\label{sec:data-pre-process}
%
 The data-set comes from Scania CV AB and it is specific to the \gls*{aps} system on the vehicle. The data-set is confidential, so the feature names have been redacted. It consists of 8 main features and the time domain is in seconds. Some of the data-set's features are discrete while others are continuous. Furthermore, the data-sets contain runs with normal behaviour, and others contain physical injected faults.
%
Figure~\ref{fig:scania-data} shows three runs performed under the same road conditions, with the first one having a normal \gls*{aps} behaviour, the second one exhibiting anomalous behaviour, and the third having the fault injected midway through the run (mixed data). Features 4, 6, and 7 are overlapping.

\def\layerfigwidth{0.8}
\def\layerfigheight{0.25}
\def\ydist{-3.7}
\def\plotlinewidth{1}
\begin{figure} [t]
    \centering
\begin{tikzpicture}[scale = 0.61]
\begin{axis} [every axis plot post/.append style={
  mark=none,smooth,-,line width = \plotlinewidth pt},
ylabel={Anomalous Data},
axis x line=bottom,
axis y line=left,
axis line style={-,line width = 1pt},
% ticks=none,
width=\layerfigwidth\textwidth,
height=\layerfigheight\textwidth,
xmax = 1000,
ymax=1.005
] 
    \addplot[fone] table [col sep=comma,header=true,x index=0,y index=1] {Data/anamolus.csv};
    \addplot[ftwo] table [col sep=comma,header=true,x index=0,y index=2] {Data/anamolus.csv};
    \addplot[fthree] table [col sep=comma,header=true,x index=0,y index=3] {Data/anamolus.csv};
    \addplot[ffour] table [col sep=comma,header=true,x index=0,y index=4] {Data/anamolus.csv};
    \addplot[ffive] table [col sep=comma,header=true,x index=0,y index=5] {Data/anamolus.csv};  
    \addplot[fsix] table [col sep=comma,header=true,x index=0,y index=6] {Data/anamolus.csv};
    \addplot[fseven] table [col sep=comma,header=true,x index=0,y index=7] {Data/anamolus.csv};
    \addplot[feight] table [col sep=comma,header=true,x index=0,y index=8] {Data/anamolus.csv};
\end{axis}

\begin{axis} [at={(0,\ydist cm)},every axis plot post/.append style={mark=none,smooth,-,line width = \plotlinewidth pt},
% xlabel={Time Step [s]},
ylabel={Normal Data},
axis x line=bottom,
axis y line=left,
axis line style={-,line width = 1pt},
% ticks=none,
width=\layerfigwidth\textwidth,
height=\layerfigheight\textwidth,
xmax = 1000,
ymax=1.005
]
    \addplot[fone] table [col sep=comma,header=true,x index=0,y index=1] {Data/normal.csv};
    \addplot[ftwo] table [col sep=comma,header=true,x index=0,y index=2] {Data/normal.csv};
    \addplot[fthree] table [col sep=comma,header=true,x index=0,y index=3] {Data/normal.csv};
    \addplot[ffour] table [col sep=comma,header=true,x index=0,y index=4] {Data/normal.csv};
    \addplot[ffive] table [col sep=comma,header=true,x index=0,y index=5] {Data/normal.csv};  
    \addplot[fsix] table [col sep=comma,header=true,x index=0,y index=6] {Data/normal.csv};
    \addplot[fseven] table [col sep=comma,header=true,x index=0,y index=7] {Data/normal.csv};
    \addplot[feight] table [col sep=comma,header=true,x index=0,y index=8] {Data/normal.csv};
\end{axis}

\begin{axis} [at={(0,\ydist*2 cm)},every axis plot post/.append style={mark=none,smooth,-,line width = \plotlinewidth pt},
xlabel={Time Step [s]},
ylabel={Mixed Data},
axis x line=bottom,
axis y line=left,
axis line style={-,line width = 1pt},
% ticks=none,
width=\layerfigwidth\textwidth,
height=\layerfigheight\textwidth,
xmax = 1000,
ymax=1.005,
legend columns=4,
legend style={at={(0.5,-0.5)},anchor=north,font=\large}
]
    \addplot[fone] table [col sep=comma,header=true,x index=0,y index=1] {Data/mix-data.csv};
    \addplot[ftwo] table [col sep=comma,header=true,x index=0,y index=2] {Data/mix-data.csv};
    \addplot[fthree] table [col sep=comma,header=true,x index=0,y index=3] {Data/mix-data.csv};
    \addplot[ffour] table [col sep=comma,header=true,x index=0,y index=4] {Data/mix-data.csv};
    \addplot[ffive] table [col sep=comma,header=true,x index=0,y index=5] {Data/mix-data.csv};  
    \addplot[fsix] table [col sep=comma,header=true,x index=0,y index=6] {Data/mix-data.csv};
    \addplot[fseven] table [col sep=comma,header=true,x index=0,y index=7] {Data/mix-data.csv};
    \addplot[feight] table [col sep=comma,header=true,x index=0,y index=8] {Data/mix-data.csv};
    \legend{Feature \#1,Feature \#2,Feature \#3,Feature \#4,Feature \#5,Feature \#6,Feature \#7,Feature \#8}
\end{axis}

\end{tikzpicture}
\caption{The real APS time-series data (anomalous vs. normal vs. mixed).}
\label{fig:scania-data}
\end{figure}

To improve the training, data are usually normalized using, e.g., Min-Max normalization. In our case, each feature can be normalized based on the known (constant) capabilities of that specific sensor. %These boundaries are constant and known for any vehicle having the same sensors. 
However, in practice, the span between the minimum and maximum values for each feature is much larger than the normal operating values and my not represent the real distribution.
%
To improve on the normalization technique, we implemented a slightly different approach. (see Fig.~\ref{fig:normalize}). 
At the beginning of the communication, all workers would send their minimum and maximum values for each feature in their local data-set to the server. The server decides on minimum and maximum values based on the lowest minimum value and the highest maximum value among all the workers for each feature. The server sends the decided min and max values to all workers to be used to normalize their local data. Note that the output from the generator will have data that is normalized based on the worker limits. That means there need be an extra step to un-normalize the data to switch it back to either normalized data based on the known capabilities of the features, or to real values.
\newcommand\normscale{0.6}
\begin{figure}[t]
    \centering
        \fcolorbox{blue}{white}{
\begin{tikzpicture}[line width = 1pt,rounded corners,
arrow/.style={line width = 2.5pt},scale=\normscale,every node/.style={scale=\normscale},font = \large]

\def\spacex{1.0}

\node[draw,fill=servcol,align=left,inner sep=10pt,label={north:\textbf{Server}}] (serv) at (0,0) {\textbf{Feature \#1:} 19 - 41 \\ \textbf{Feature \#2:} 31  - 66};
\node[inner sep=0pt,below = 1.8 of serv] (truck2) {\includegraphics[width=.2\textwidth]{Images/scania-truck.png}};

\node[inner sep=0pt,left = \spacex of truck2] (truck1) {\includegraphics[width=.2\textwidth]{Images/scania-truck.png}};
\node[inner sep=0pt, right = \spacex of truck2] (truck3) {\includegraphics[width=.2\textwidth]{Images/scania-truck.png}};

\node[draw, above left = 0 and 0 of truck1,align=left,anchor = north west,fill=white,fill opacity = 0.7,fill=workcol] (work1) {\textbf{Feature \#1:} 19 - 36 \\ \textbf{Feature \#2:} 45  - 66};

\node[draw, above left = 0 and 0 of truck2,align=left,anchor = north west,fill=white,fill opacity = 0.7,fill=workcol] (work2) {\textbf{Feature \#1:} 23 - 29\\\textbf{Feature \#2:}  31 - 36};


\node[draw, above left = 0 and 0 of truck3,align=left,anchor = north west,fill=white,fill opacity = 0.7,fill=workcol] (work3) {\textbf{Feature \#1:} 26 - 41 \\ \textbf{Feature \#2:} 40 - 55};

\draw[arrow,myred] (serv.180) to [bend right] node[above,sloped,align=center] {19 - 41 \\ 31  - 66} (work1.100);
\draw[arrow] (work1.80) to node[below,sloped,align=center] {19 - 36 \\ 45  - 66} (serv.185);

\draw[arrow,myred] (serv.0) to [bend left] node[above,sloped,align=center] {19 - 41 \\ 31  - 66} (work3.80);
\draw[arrow] (work3.100) to node[below,sloped,align=center] {26 - 41 \\ 40  - 55} (serv.-5);

\draw[arrow] (work2.80) to node[above,sloped,align=center] {23 - 29 \\ 31  - 36} (serv.280);
\draw[arrow,myred] (serv.260) to [bend right] node[above,rotate=180,sloped,align=center] {19 -41 \\ 31 - 66} (work2.100);

\end{tikzpicture}}
\textcolor{blue}{\marginnote{AR12 AR21}}
    \caption{\textcolor{blue}{Decentralized data normalization based on workers' minimum and maximum feature values.}}
    \label{fig:normalize}
\end{figure}

\subsection{Decentralized Training Sequence}
%
The full training sequence of the decentralized time-series \gls*{gan} model involves the following steps:
%
\begin{itemize}
\item For each epoch, a target sequence width is determined in order to calculate the necessary noise size and actual sequence width output from the generator.
\item Random noise is generated and used to create fake sequences, which are then sent to all workers.
\item Workers create samples from their local real data that match the sequence width of the fake data received.
\item During each iteration, workers feed their discriminator a real sample with a target of one, and then feed it a fake sample with a target of zero.
\item Loss is calculated for both the real and fake data using MSE loss. The overall loss for each discriminator is the average of both losses.
\item The overall loss is back-propagated to improve each discriminator using Adam optimizer~\cite{adam_optimizer}.
\item The fake losses for all workers are compared, and based on the training strategy either:
\begin{itemize}
\item The worker with the highest/lowest loss is chosen.
\item The weighed average of all discriminator parameters is used to create a global discriminator. % that overrides all the discriminators at each iteration.
\end{itemize}
\item The chosen worker's discriminator or the global discriminator's output on the fake data is used to calculate the loss of the generator with a target of one.
\item The generator loss is used to back-propagate and improve the generator using Adam optimizer.
\item Steps repeat with a different target sequence width.
\end{itemize}


\subsection{Evaluation Metrics}
%
To quantitatively evaluate the time-series output, we used an autoregressive model. Autoregression is often used to analyze and forecast time-series data, and can be used as a metric for evaluating the performance of a forecasting model~\cite{rnntimeseries}. For \gls*{gan}, this can be done in 3 different ways: %. Figure \ref{fig:trtr} visualizes those methods and they are as follows:
\begin{itemize}
    \item \textit{\gls*{trtr}:} Using only real data is useful to establish a baseline that determines how well the model is able to capture the statistical properties of the real data. This can then be used to validate the performance of the other methods.
    \item \textit{\gls*{trts}:} Instead of testing on real data, the synthetic data generated by the 
    \gls*{gan} model is used. This checks if the \gls*{gan} model has learned the underlying structure of the real data effectively.
    \item \textit{\gls*{tstr}:} This is opposite of the previous method and will be used as a secondary metric. The method determines if the synthetic data generated by the \gls*{gan} is of sufficient quality to be used in place of real data for training purposes.
\end{itemize}

%In our qualitative evaluation, we visually inspected the sequence output and further employed the use of \gls*{pca} by calculating it for both the real and synthetic data to visualize if the \gls*{gan} model has accurately captured the real data distribution.


%\def\trtrscale{0.6}
%\begin{table}[t]
\centering
\caption{TRTR results for each data behaviour}
\label{tab:trtr-main}
% \resizebox{\columnwidth}{!}{%

\begin{tabular}{lccc}
\Xhline{1.5pt}
\multicolumn{1}{c}{\multirow{2}{*}{\sc{Data Type}}} & \multicolumn{3}{c}{\sc{TRTR}} \\
\multicolumn{1}{c}{} & \sc{R2} & \sc{MAE} & \sc{RMSE} \\ \Xhline{1.2pt}
Anomalous  & 0.588       & 0.062        & 0.165         \\ \hline
Normal     & 0.819       & 0.039        & 0.104         \\ \hline
Mixed      & 0.530       & 0.071        & 0.172         \\\Xhline{1.5pt}
\end{tabular}%
% }
\end{table}
