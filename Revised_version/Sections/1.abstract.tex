\begin{abstract}

% This is very abstract. See \href{https://ieee-itss.org/pub/t-iv/}{info}
% The regular paper is 10 pages the short is 6. I think we can try 6 pages, but if we realize it is too difficult to compress the information, we can move to 10 pages.

Ensuring the privacy of personal data is crucial in the era of big data, especially in the transportation industry where sensitive data needs to be processed to develop intelligent vehicle technologies. In particular, collecting and analyzing anomalous data is essential for improving vehicle safety and performance, but accessing such data is often difficult and costly. To address this problem, we propose a novel approach to generating synthetic anomalous data using \glspl*{gan} and \gls*{fl}. By training the \glspl*{gan} on data generated from multiple vehicles, we can develop models without accessing the raw data, thus preserving the privacy of personal information. Our approach involves a \gls*{cnn}-based architecture for both the generator and discriminator, with the generator residing on the server and a separate discriminator at each vehicle. This design reduces the computational demand on edge devices and enables us to train the \glspl*{gan} using \gls*{fl}. We experiment with different \gls*{fl} strategies and find that the best performer favored the least forgiving discriminator considering data from a pool of vehicles. Our results demonstrate the feasibility of using \gls*{fl} with CNN-based \glspl*{gan} to generate synthetic time-series data for training models in a privacy-preserving manner. This approach has potential applications in the transportation industry, particularly in the context of intelligent vehicles and automated driving systems.


\end{abstract}

\begin{IEEEkeywords}
Generative Adversarial Network, Federated Learning, Vehicle Data, Time-Series
\end{IEEEkeywords}